\chapter{Identifying Features using IPFs}
\label{chap:featureid}

%##################################################################################################
\section{Chapter Overview}
%##################################################################################################

In the previous chapter, we saw how to construct image partition forests from medical images using the watershed and waterfall transforms from mathematical morphology. This chapter now illustrates how these partition forests can be used when attempting to automatically identify features in abdominal CT scans.

Before looking at how to identify specific features in detail, I first describe a number of techniques that can be used when constructing new feature identifiers. These range from simple techniques, such as filtering the branch nodes of the IPF based on a predicate (i.e.~searching for branch nodes that satisfy certain properties), to more involved and novel techniques such as \emph{stratified region growing}. I then describe two different classes of feature identifier: (1) those that identify features in 2D axial (X-Y) CT slices, as were implemented in my trial system, \emph{centipede}, and described in papers such as \cite{gvccimi08} and \cite{gvcispa09}, and (2) those that identify features in 3D CT volumes, as have been implemented in my final system, \emph{millipede}. (The approaches needed in the two cases have a certain amount in common, but differ in the details: for instance, when identifying the spine in 3D volumes, it is possible to use the property that it is a feature that extends through all the axial slices. This is quite powerful, since few 3D regions will do so. The same approach evidently cannot be used in 2D, since there is only one slice.)

This lays the foundation for Chapter~\ref{chap:validation}, in which the 3D feature identifiers used in \emph{millipede} will be validated against manually-produced, `gold-standard' results to quantify their accuracy. (The same will not be done for the 2D identifiers in \emph{centipede}, as they have now been superseded by their 3D counterparts; however, the quality of the 2D spine results was categorised to some extent in \cite{gvcispa09}.)

%##################################################################################################
\section{General Techniques}
%##################################################################################################

TODO

%Owing to the fact that different features in abdominal CT scans have widely differing properties, it is difficult to construct a framework to enable new feature identifiers to be constructed in anything like a mechanical way -- in general, a certain amount of invention is still required. However, there is nevertheless a great deal of low-level commonality between different identifiers -- while they may not share the same higher-level structure, they all make use of a number of general techniques, e.g.~filtering for regions that satisfy particular properties. This section thus describes these `building blocks', as a necessary precursor to presenting the actual identification algorithms.

%################################################
\subsection{Branch Node Filtering}
%################################################

TODO

%---
\begin{stulisting}[p]
\caption{Branch Node Filtering Implementation}
\label{code:featureid-techniques-branchnodefiltering}
\lstinputlisting[style=Default]{featureid/featureid-techniques-branchnodefiltering.lst}
\end{stulisting}
%---

%################################################
\subsection{Bayesian Classification}
%################################################

TODO

%################################################
\subsection{Conditional Morphology}
%################################################

TODO

%---
\begin{stulisting}[p]
\caption{Implementation of Conditional Morphological Operators}
\label{code:featureid-techniques-conditionalmorphology}
\lstinputlisting[style=Default]{featureid/featureid-techniques-conditionalmorphology.lst}
\end{stulisting}
%---

%################################################
\subsection{Stratified Region Growing}
%################################################

TODO

%---
\begin{stulisting}[p]
\caption{Stratified Region Growing Implementation}
\label{code:featureid-techniques-stratifiedregiongrowing}
\lstinputlisting[style=Default]{featureid/featureid-techniques-stratifiedregiongrowing.lst}
\end{stulisting}
%---

%##################################################################################################
\section{Feature Identification in Axial Slices}
%##################################################################################################

TODO

%##################################################################################################
\section{Feature Identification in 3D Volumes}
%##################################################################################################

TODO

%---
\begin{stulisting}[p]
\caption{Spine Identification in 3D}
\label{code:featureid-3d-spineidentification}
\lstinputlisting[style=Default]{featureid/featureid-3d-spineidentification.lst}
\end{stulisting}
%---

%---
\begin{stulisting}[p]
\caption{Spinal Cord Identification in 3D}
\label{code:featureid-3d-spinalcordidentification}
\lstinputlisting[style=Default]{featureid/featureid-3d-spinalcordidentification.lst}
\end{stulisting}
%---

%##################################################################################################
\section{Chapter Summary}
%##################################################################################################

TODO
