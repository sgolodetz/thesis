\chapter{Identifying Features using IPFs}
\label{chap:featureid}

%---
\section{Chapter Overview}

TODO: This needs a rewrite.

In the previous chapter, we saw how image partition forests (IPFs), as introduced in Chapter~\ref{chap:ipfs}, could be constructed from CT image slices or volumes by adapting the morphological watershed and waterfall transforms. This chapter describes the algorithms I have developed for performing automatic feature identification using the constructed IPFs. The validity and usefulness of these algorithms will be discussed in Chapter~\ref{chap:assessment}.

The ultimate goal of the feature identification algorithms described in this chapter is to identify organs in the images as automatically as possible. One criterion used by radiologists \cite{?} when identifying organs is their location in the image, and this localization information is equally helpful to tools attempting to automate the process. For this reason, it is extremely helpful to establish a frame of reference (or coordinate system) for the image against which organ location can be measured. It is sensible to base this on rigid structures (such as bone) that do not compress when the patient moves. For this reason, the first algorithms described in this chapter are for automatic spine and rib identification. The process of generating a suitable reference frame from the identified spine and ribs is then illustrated, followed by algorithms to actually identify the organs of interest.

%---
\section{Automatic Rib Identification}

TODO

%---
\section{Automatic Spine Identification}

TODO

%---
\section{Reference Frame Generation}

TODO

%---
\section{Automatic Organ Identification}

TODO

%---
\section{Applications of Feature Identification}
\label{sec:featureid-applications}

TODO

\subsection{Volume Visualization}

TODO

\subsection{Volume Estimation}

TODO

%---
\section{Chapter Summary}

TODO: This needs a rewrite.

In this chapter, we saw a possible approach to automatically identifying organs in medical image volumes. The next chapter will discuss the validation issues involved, compare the approach to existing approaches used, and assess the contribution of the algorithms.
