\begin{abstract}
This thesis is about how to allow a computer to identify important features (such as major organs) in computerised tomography (CT) scans of the abdomen in a relatively automatic way, whilst still making it easy for users to interact with and manipulate the results. Identifying such features is important if they are to be visualized in 3D for the purposes of diagnosis or surgical planning, or if their volumes are to be calculated when assessing a patient's response to therapy, but to identify them manually is a time-consuming, error-prone and somewhat tedious task. Some degree of automation using a computer is therefore highly desirable, and indeed a small number of existing approaches have even attempted to fully automate the process of identifying multiple abdominal organs at once. However, no existing method is capable of achieving results that are completely accurate in all cases, and due to the difficulties even of specifying when a result is correct, the development of such a method does not seem likely in the near future. For this reason, it is important that medics retain the ability to manipulate the results and correct them when automated methods fail.

The research described here thus proposes a way of facilitating both relatively automatic feature identification and intuitive editing of the results by representing CT images as a hierarchy of partitions, or \emph{image partition forest} (IPF). This data structure has appeared extensively in existing literature, under a variety of different names, but its potential uses for editing have hitherto received little attention. This thesis shows how it can be used for this purpose, by presenting a rich set of algorithms that allow the user to modify the IPF, and select and identify features therein, in a intuitive way via a graphical user interface (GUI). It further shows how such an IPF can be initially constructed from a set of CT images using techniques from mathematical morphology, before presenting a series of novel methods for automatic feature identification in both 2D axial CT slices and 3D CT volumes.

All of the techniques described have been implemented in one of two systems developed to illustrate their use. TODO
\end{abstract}
