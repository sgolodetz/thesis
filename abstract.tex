\begin{abstract}
This thesis is about how to allow a computer to identify important features (such as major organs) in computerised tomography (CT) scans of the abdomen in a relatively automatic way, whilst still making it easy for users to interact with and manipulate the results. Identifying such features is important if they are to be visualized in 3D for the purposes of diagnosis or surgical planning, or if their volumes are to be calculated when assessing a patient's response to therapy, but to identify them manually is a time-consuming, error-prone and somewhat tedious task. Some degree of automation using a computer is therefore highly desirable, and indeed a small number of existing approaches have even attempted to fully automate the process of identifying multiple abdominal organs at once. However, no existing method is capable of achieving results that are completely accurate in all cases, and due to the difficulties even of specifying when a result is correct, the development of such a method does not seem likely in the near future. For this reason, it is important that medics retain the ability to manipulate the results and correct them when automated methods fail.

TODO
\end{abstract}
