\chapter[Partition Forests -- Support Algorithms]{Partition Forests --\\Support Algorithms}
\label{chap:appendixpf}

\index{partition forests!support algorithms|(}

%##################################################################################################
\section{The Forest Data Structure}
%##################################################################################################

%################################################
\paragraph{Ancestor Of}
%################################################

\begin{stulisting}[H]
\caption{Forest : Ancestor Of Implementation}
\begin{lstlisting}[style=Default]
function ancestor_of
:	(node : NodeID; layerIndex : int) $\to$ NodeID

	if layerIndex < node.layer() or layerIndex > highest_layer() then throw;

	var cur : NodeID := node;
	while cur.layer() < layerIndex
		var layer : IForestLayer := forest_layer(cur.layer());
		cur := NodeID(cur.layer() + 1, layer.node_parent(cur.index());
	return cur;
\end{lstlisting}
\end{stulisting}

%################################################
\paragraph{Are Connected}
%################################################

\begin{stulisting}[H]
\caption{Forest : Are Connected Implementation}
\begin{lstlisting}[style=Default]
function are_connected
:	(nodes : Set<int>; layerIndex : int) $\to$ bool

	if nodes.empty() then return false;

	find_connected_component(ref nodes, layerIndex);

	// If no nodes remain after finding the first connected component, the nodes must
	// be connected, and vice-versa.
	return nodes.empty();
\end{lstlisting}
\end{stulisting}

%################################################
\paragraph{Children Of}
%################################################

\begin{stulisting}[H]
\caption{Forest : Children Of Implementation}
\begin{lstlisting}[style=Default]
function children_of
:	(node : NodeID) $\to$ Set<NodeID>

	// Special case: nodes in the leaf layer have no children.
	if node.layer() = 0 then return {};

	var layer : BranchLayer? := checked_branch_layer(node.layer());
	if layer = null or not layer.has_node(node.index()) then throw;

	var result : Set<NodeID>;
	var childLayerIndex : int := node.layer() - 1;
	for each c : int $\in$ layer.node_children(node.index());
		result.insert(NodeID(childLayerIndex, c));
	return result;
\end{lstlisting}
\end{stulisting}

%################################################
\paragraph{Find Common Ancestor Layer}
%################################################

\begin{stulisting}[H]
\caption{Forest : Find Common Ancestor Layer Implementation}
\label{code:appendixpf-findcommonancestorlayer}
\begin{lstlisting}[style=Default]
function find_common_ancestor_layer
:	(component : Set<int>; layerIndex : int) $\to$ int

	var curs : ref Set<int> := ref component;
	while layerIndex $\le$ highest_layer() and curs.size() > 1
		var layer : ForestLayer := forest_layer(layerIndex);
		var parents : Set<int>;
		for each n : int $\in$ curs
			parents.insert(layer.node_parent(n));
		curs := ref parents;
		layerIndex := layerIndex + 1;
	return layerIndex;
\end{lstlisting}
\end{stulisting}

%################################################
\paragraph{Find Common Ancestor Layer And New Chain}
%################################################

\begin{stulisting}[H]
\caption{Forest : Find Common Ancestor Layer And New Chain Implementation}
\label{code:appendixpf-findcommonancestorlayerandnewchain}
\begin{lstlisting}[style=Default]
function find_common_ancestor_layer_and_new_chain
:	(oldParent : int; newParent : int; layerIndex : int) $\to$ (int,Chain)

	var newChain : Chain;
	var curOld, curNew : int := oldParent, newParent;
	while layerIndex $\le$ highest_layer() and curOld $\ne$ curNew
		newChain.push_front(NodeID(layerIndex, curNew));
		var layer : ForestLayer := forest_layer(layerIndex);
		curOld := layer.node_parent(curOld);
		curNew := layer.node_parent(curNew);
		layerIndex := layerIndex + 1;
	return (layerIndex, newChain);
\end{lstlisting}
\end{stulisting}

%################################################
\paragraph{Find Connected Component}
%################################################

\begin{stulisting}[H]
\caption{Forest : Find Connected Component Implementation}
\begin{lstlisting}[style=Default]
function find_connected_component
:	(nodes : ref Set<int>; layerIndex : int) $\to$ Set<int>

	var result : Set<int>;
	var layer : ForestLayer := forest_layer(layerIndex);
	var seed : int := lowest_int(nodes);
	nodes.erase(seed);

	var q : Queue<int>;
	q.push(seed);
	while not q.empty()
		var n : int := q.pop();
		result.insert(n);
		for each e : Edge $\in$ layer.adjacent_edges(n)
			var other : int := other_end(e, n);
			if nodes.contains(other) then
				q.push(other);
				nodes.erase(other);
	return result;
\end{lstlisting}
\end{stulisting}

%################################################
\paragraph{Find Connected Components}
%################################################

\begin{stulisting}[H]
\caption{Forest : Find Connected Components Implementation}
\begin{lstlisting}[style=Default]
function find_connected_components
:	(nodes : Set<int>; layerIndex : int) $\to$ Vector<Set<int>>

	var result : Vector<Set<int>>;
	while not nodes.empty()
		result.push_back(find_connected_component(nodes, layerIndex));
	return result;
\end{lstlisting}
\end{stulisting}

%################################################
\paragraph{Receptive Region Of}
%################################################

\begin{stulisting}[H]
\caption{Forest : Receptive Region Of Implementation}
\begin{lstlisting}[style=Default]
function receptive_region_of : (node : NodeID) $\to$ Deque<int>
	var MARKER : int := -1;
	var layerIndex : int := node.layer();
	var nodeIndices : Deque<int> := [node.index(), MARKER];
	while layerIndex $>$ 0
		var layer : BranchLayer := branch_layer(layerIndex);
		while true	// iterate until a break occurs
			var n : int := nodeIndices.front();
			nodeIndices.pop_front();
			if n = MARKER then break;
			nodeIndices.append(layer.node_children(n));
		nodeIndices.push_back(MARKER);
		layerIndex := layerIndex - 1;
	nodeIndices.pop_back();
	return nodeIndices;
\end{lstlisting}
\end{stulisting}

%##################################################################################################
\section{The Selection Data Structure}
%##################################################################################################

%################################################
\paragraph{Clear}
%################################################

\begin{stulisting}[H]
\caption{Selection : Clear Implementation}
\begin{lstlisting}[style=Default]
function clear_impl : () $\to$ Modification
	var modification : Modification;
	for i := 0 up to forest.highest_layer()
		for each n $\in$ layer(i)
			modification.erase_node(NodeID(i, n));
		layer(i).clear();
	listeners.selection_was_cleared();
	return modification;
\end{lstlisting}
\end{stulisting}

%################################################
\paragraph{Combine}
%################################################

\begin{stulisting}[H]
\caption{Selection : Combine Implementation}
\begin{lstlisting}[style=Default]
function combine : (lhs : PartitionForestSelection; rhs : PartitionForestSelection) $\to$ $\emptyset$
	assert empty();	// only invoke this method on newly-created selections
	this := lhs;
	for each n $\in$ rhs.nodes()
		select_node_impl(n);
\end{lstlisting}
\end{stulisting}

%################################################
\paragraph{Combine Using Leaves}
%################################################

\begin{stulisting}[H]
\caption{Selection : Combine Using Leaves Implementation}
\begin{lstlisting}[style=Default]
function combine_using_leaves
:	(lhs : PartitionForestSelection; rhs : PartitionForestSelection) $\to$ $\emptyset$
	assert empty();	// only invoke this method on newly-created selections
	for each n $\in$ lhs.nodes()
		layer(0).append(forest().receptive_region_of(n));
	for each n $\in$ rhs.nodes()
		layer(0).append(forest().receptive_region_of(n));

	consolidate_all();
\end{lstlisting}
\end{stulisting}

%################################################
\paragraph{Consolidate All}
%################################################

\begin{stulisting}[H]
\caption{Selection : Consolidate All Implementation}
\begin{lstlisting}[style=Default]
function consolidate_all : () $\to$ $\emptyset$
	for i : int := 0 up to forest().highest_layer() - 1
		var parents : Set<NodeID>;
		for each n $\in$ layer(i)
			parents.insert(forest().parent_of(NodeID(i,n)));
		for each p $\in$ parents
			consolidate_node(p, null);
\end{lstlisting}
\end{stulisting}

%################################################
\paragraph{Consolidate Node}
%################################################

\begin{stulisting}[H]
\caption{Selection : Consolidate Node Implementation}
\begin{lstlisting}[style=Default]
function consolidate_node : (node : NodeID; modification : ref Modification?) $\to$ bool
	// Check to see if all the children of the specified node are selected.
	for each c $\in$ forest().children_of(node)
		if not in_representation(c) then return false;

	// If they are, deselect them and select this node instead.
	for each c $\in$ forest().children_of(node)
		erase_node(c, ref modification);
	insert_node(node, ref modification);

	listeners.node_was_consolidated(node);
	return true;
\end{lstlisting}
\end{stulisting}

%################################################
\paragraph{Consolidate Upwards From Node}
%################################################

\begin{stulisting}[H]
\caption{Selection : Consolidate Upwards From Node Implementation}
\begin{lstlisting}[style=Default]
function consolidate_upwards_from_node
:	(node : NodeID; modification : ref Modification?) $\to$ $\emptyset$
	var cur : NodeID := node;
	while not cur.invalid() and consolidate_node(cur, ref modification)
		cur := forest().parent_of(cur);
\end{lstlisting}
\end{stulisting}

%################################################
\paragraph{Contains}
%################################################

\begin{stulisting}[H]
\caption{Selection : Contains Implementation}
\begin{lstlisting}[style=Default]
function contains : (node : NodeID) $\to$ bool
	if in_representation(node) then return true;
	var trail : Stack<NodeID>;
	return not find_ancestor_in_representation(node, trail).invalid();
\end{lstlisting}
\end{stulisting}

%################################################
\paragraph{Deconsolidate Node}
%################################################

\begin{stulisting}[H]
\caption{Selection : Deconsolidate Node Implementation}
\begin{lstlisting}[style=Default]
function deconsolidate_node : (node : NodeID; modification : ref Modification?) $\to$ $\emptyset$
	// Replace the selected node with its children in the forest.
	erase_node(node, ref modification);
	for each c $\in$ forest().children_of(node)
		insert_node(c, ref modification);
	listeners.node_was_deconsolidated(node);
\end{lstlisting}
\end{stulisting}

%################################################
\paragraph{Descendants In Layer}
%################################################

\begin{stulisting}[H]
\caption{Selection : Descendants In Layer Implementation}
\begin{lstlisting}[style=Default]
function descendants_in_layer : (node : NodeID; layerIndex : int) $\to$ List<NodeID>
	assert node.layer() $>$ layerIndex;

	var children : Set<NodeID> := forest().children_of(node);
	if node.layer() $\ne$ layerIndex + 1 then
		var descendants : List<NodeID>;
		for each c $\in$ children
			descendants.append(descendants_in_layer(c, layerIndex));
		return descendants;
	else return children;
\end{lstlisting}
\end{stulisting}

%################################################
\paragraph{Descendants In Representation}
%################################################

\begin{stulisting}[H]
\caption{Selection : Descendants In Representation Implementation}
\begin{lstlisting}[style=Default]
function descendants_in_representation : (node : NodeID) $\to$ List<NodeID>
	var descendants : List<NodeID>;
	for each c $\in$ forest().children_of(node)
		if in_representation(c) then descendants.push_back(c);
		else descendants.append(descendants_in_representation(c));
	return descendants;
\end{lstlisting}
\end{stulisting}

%################################################
\paragraph{Erase Node}
%################################################

\begin{stulisting}[H]
\caption{Selection : Erase Node Implementation}
\begin{lstlisting}[style=Default]
function erase_node : (node : NodeID; modification : ref Modification?) $\to$ $\emptyset$
	layer(node.layer()).erase(node.index());
	if modification $\ne$ null then modification.erase_node(node);
\end{lstlisting}
\end{stulisting}

%################################################
\paragraph{Find Ancestor In Representation}
%################################################

\begin{stulisting}[H]
\caption{Selection : Find Ancestor In Representation Implementation}
\begin{lstlisting}[style=Default]
function find_ancestor_in_representation
:	(node : NodeID; trail : ref Stack<NodeID>) $\to$ NodeID
	var parent : NodeID := forest().parent_of(node);
	trail.push(parent);

	// If this node has no parent, then it definitely has no ancestor in the
	// representation.
	if parent.invalid() then return parent;

	// If this node has a parent and it's in the representation, then return it.
	if in_representation(parent) then return parent;

	// Otherwise, any ancestor this node may have is also its parent's ancestor,
	// so recurse.
	return find_ancestor_in_representation(parent, trail);
\end{lstlisting}
\end{stulisting}

%################################################
\paragraph{Insert Node}
%################################################

\begin{stulisting}[H]
\caption{Selection : Insert Node Implementation}
\begin{lstlisting}[style=Default]
function insert_node : (node : NodeID; modification : ref Modification?) $\to$ $\emptyset$
	layer(node.layer()).insert(node.index());
	if modification $\ne$ null then modification.insert_node(node);
\end{lstlisting}
\end{stulisting}

%################################################
\paragraph{Merge Layer}
%################################################

\begin{stulisting}[H]
\caption{Selection : Merge Layer Implementation}
\begin{lstlisting}[style=Default]
/*
Calculates the layer in which any merging of the selected nodes should happen.

viewLayer - The layer being `viewed' by the user.
*/
function merge_layer : (viewLayer : int) $\to$ int
	for i : int := 0 up to viewLayer - 1
		if not layer(i).empty() return i;
	return viewLayer;
\end{lstlisting}
\end{stulisting}

%################################################
\paragraph{Replace With Selection}
%################################################

\begin{stulisting}[H]
\caption{Selection : Replace With Selection Implementation}
\begin{lstlisting}[style=Default]
function replace_with_selection_impl
:	(selection : PartitionForestSelection) $\to$ Modification
	var modification : Modification;
	for i : int := 0 up to forest().highest_layer()
		for each n $\in$ layer(i)
			modification.erase_node(NodeID(i, n));
		layer(i).clear();
	for each n $\in$ selection
		modification.insert_node(n);
		layer(n.layer()).insert(n.index());
	listeners.selection_was_replaced(selection);
	return modification;
\end{lstlisting}
\end{stulisting}

%################################################
\paragraph{Split Selection}
%################################################

\begin{stulisting}[H]
\caption{Selection : Split Selection Implementation}
\begin{lstlisting}[style=Default]
function split_selection
:	(trail : ref Stack<NodeID>; modification : ref Modification) $\to$ $\emptyset$
	while not trail.empty()
		var cur : NodeID := trail.top();
		trail.pop();
		deconsolidate_node(cur, ref modification);
\end{lstlisting}
\end{stulisting}

%################################################
\paragraph{Subtract}
%################################################

\begin{stulisting}[H]
\caption{Selection : Subtract Implementation}
\begin{lstlisting}[style=Default]
function subtract : (lhs : PartitionForestSelection; rhs : PartitionForestSelection) $\to$ $\emptyset$
	assert empty();	// only invoke this method on newly-created selections
	this := lhs;
	for each n $\in$ rhs.nodes()
		deselect_node_impl(n);
\end{lstlisting}
\end{stulisting}

%################################################
\paragraph{Subtract Using Leaves}
%################################################

\begin{stulisting}[H]
\caption{Selection : Subtract Using Leaves Implementation}
\begin{lstlisting}[style=Default]
function subtract_using_leaves
:	(lhs : PartitionForestSelection; rhs : PartitionForestSelection) $\to$ $\emptyset$
	assert empty();	// only invoke this method on newly-created selections
	for each n $\in$ lhs.nodes()
		layer(0).append(forest().receptive_region_of(n));
	for each n $\in$ rhs.nodes()
		layer(0).erase_all(forest().receptive_region_of(n));
	consolidate_all();
\end{lstlisting}
\end{stulisting}

%################################################
\paragraph{Node Iterators}
%################################################

\begin{stulisting}[H]
\caption{Selection : Node Iterators Implementation}
\lstinputlisting[style=Default]{appendixpf/appendixpf-selection-nodeiterators.lst}
\end{stulisting}

\newpage

%################################################
\paragraph{View At Layer Iterators}
%################################################

\begin{stulisting}[H]
\caption{Selection : View At Layer Iterators Implementation}
\lstinputlisting[style=Default]{appendixpf/appendixpf-selection-viewatlayeriterators.lst}
\end{stulisting}

\newpage

%##################################################################################################
\section{The Multi-Feature Selection Data Structure}
%##################################################################################################

%################################################
\paragraph{Clear All}
%################################################

\begin{stulisting}[H]
\caption{Multi-Feature Selection : Clear All Implementation}
\begin{lstlisting}[style=Default]
function clear_all : () $\to$ $\emptyset$
	"begin_command_sequence();" // handled by the command sequence guard
	for each (feature,ref selection) $\in$ mfs
		selection.clear();
	"end_command_sequence();" // handled by the command sequence guard
\end{lstlisting}
\end{stulisting}

%################################################
\paragraph{Clear Feature}
%################################################

\begin{stulisting}[H]
\caption{Multi-Feature Selection : Clear Feature Implementation}
\begin{lstlisting}[style=Default]
function clear_feature : (feature : Feature) $\to$ $\emptyset$
	if has_selection(feature) then
		selection(feature).clear();
\end{lstlisting}
\end{stulisting}

%################################################
\paragraph{Combine}
%################################################

\begin{stulisting}[H]
\caption{Multi-Feature Selection : Combine Implementation}
\lstinputlisting[style=Default]{appendixpf/appendixpf-mfs-combine.lst}
\end{stulisting}

%################################################
\paragraph{Features Of}
%################################################

\begin{stulisting}[H]
\caption{Multi-Feature Selection : Features Of Implementation}
\begin{lstlisting}[style=Default]
function features_of : (node : NodeID) $\to$ Vector<Feature>
	var features : Vector<Feature>;
	for each (feature,selection) $\in$ selections()
		if selection.contains(node) then
			features.push_back(feature);
	return features;
\end{lstlisting}
\end{stulisting}

%################################################
\paragraph{Identify Multi-Feature Selection}
%################################################

\begin{stulisting}[H]
\caption{Multi-Feature Selection : Identify Multi-Feature Selection Implementation}
\begin{lstlisting}[style=Default]
function identify_multi_feature_selection
:	(mfs : PartitionForestMultiFeatureSelection) $\to$ $\emptyset$
	"begin_command_sequence();" // handled by the command sequence guard
	for each (feature,selection) $\in$ mfs
		identify_selection(selection, feature);
	"end_command_sequence();" // handled by the command sequence guard
\end{lstlisting}
\end{stulisting}

%################################################
\paragraph{Identify Node}
%################################################

\begin{stulisting}[H]
\caption{Multi-Feature Selection : Identify Node Implementation}
\begin{lstlisting}[style=Default]
function identify_node : (node : NodeID; feature : Feature) $\to$ $\emptyset$
	selection(feature).select_node(node);
\end{lstlisting}
\end{stulisting}

%################################################
\paragraph{Identify Selection}
%################################################

\begin{stulisting}[H]
\caption{Multi-Feature Selection : Identify Selection Implementation}
\begin{lstlisting}[style=Default]
function identify_selection
:	(selection : PartitionForestSelection; feature : Feature) $\to$ $\emptyset$
	var oldSelection : ref PartitionForestSelection := selection(feature);
	var newSelection : PartitionForestSelection := make_selection(forest());
	newSelection.combine_using_leaves(oldSelection, selection);
	oldSelection.replace_with_selection(newSelection);
\end{lstlisting}
\end{stulisting}

%################################################
\paragraph{Subtract}
%################################################

\begin{stulisting}[H]
\caption{Multi-Feature Selection : Subtract Implementation}
\lstinputlisting[style=Default]{appendixpf/appendixpf-mfs-subtract.lst}
\end{stulisting}

%################################################
\paragraph{Unidentify Node}
%################################################

\begin{stulisting}[H]
\caption{Multi-Feature Selection : Unidentify Node Implementation}
\begin{lstlisting}[style=Default]
function unidentify_node : (node : NodeID; feature : Feature) $\to$ $\emptyset$
	selection(feature).deselect_node(node);
\end{lstlisting}
\end{stulisting}

%################################################
\paragraph{Unidentify Selection}
%################################################

\begin{stulisting}[H]
\caption{Multi-Feature Selection : Unidentify Selection Implementation}
\begin{lstlisting}[style=Default]
function unidentify_selection
:	(selection : PartitionForestSelection; feature : Feature) $\to$ $\emptyset$
	var oldSelection : ref PartitionForestSelection := selection(feature);
	var newSelection : PartitionForestSelection := make_selection(forest());
	newSelection.subtract_using_leaves(oldSelection, selection);
	oldSelection.replace_with_selection(newSelection);
\end{lstlisting}
\end{stulisting}

\index{partition forests!support algorithms|)}
