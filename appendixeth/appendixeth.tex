\chapter{Ethics Considerations}

When undertaking any research in fields that deal with real people (like medical image analysis), it is vitally important to consider the ethical issues involved. In the case of my research, the major issues were as follows:
%
\begin{enumerate}

\item Were the patients content for their images to be used for research purposes?
\item How could patient confidentiality be assured?
\item Where should the images be stored?

\end{enumerate}
%
It is common practice for the first issue to be handled by a process of \emph{implied consent} -- patients are informed (generally via a large sign) when attending any radiology department for their scans that their images may be used for research purposes unless they explicitly request otherwise. We were then only given images of patients who had not expressed an objection.

Patient confidentiality was ensured by the hospital carefully anonymising the images (and directory files) to remove any meta-data that could be used to identify the patient, such as the patient's name, history and birth date, and the names of the hospital, the physician reading the study, the scanner operator and the referring physician. At our end, the data was also run through our own anonymization tools (which I was advised to develop to be doubly-safe), which including assigning our own computing laboratory-specific patient IDs to make it even harder to track images back to patients. This also had the helpful side-effect of making our data organisation easier.

In terms of data storage, the data CDs as received from the hospital were stored securely in the computing laboratory, and only the further anonymized data was used for research purposes. This was stored on password-protected computers, but was in any case completely untraceable by this stage of the anonymization process.
