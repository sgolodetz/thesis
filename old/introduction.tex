\chapter*{Acknowledgements}

\thispagestyle{empty} % Remove the page number on the acknowledgement page

I would like to thank my DPhil supervisors, Dr. Stephen Cameron and Dr. Irina Voiculescu, for their guidance, patience and understanding throughout my doctorate.

I would also like to thank Dr. Andrew Protheroe (Consultant Oncologist), Dr. Zoe Traill (Consultant Radiologist), Mr. Mark Sullivan (Consultant Urological Surgeon), Mr. David Cranston (Consultant Urological Surgeon), Mr. Anthony McIntyre (Superintendent Radiographer), Dr. Nilay Patel (Specialist Registrar) and Dr. Rob Ritchie (Academic Clinical Fellow in Urology) of the Churchill Hospital, Oxford, for their invaluable help and support, and Professor Sir Michael Brady of the Oxford University Engineering Department for his help in putting us in touch with them.

Finally, I would like to thank Clare Chennells, Sue Royall, Rosemarie Llewellyn, Rachel Shaw, Carolyn Nelson and Louise Simpson for their help in organising our meetings at the Churchill.

\chapter{Introduction}

\setcounter{page}{1} % Start numbering pages from 1 again

%TODO: Set the scene and problem statement. Introduce structure of thesis, state contributions (3-5).

Despite major and prolonged research efforts around the world, cancer remains one of the leading causes of human mortality in the 21st century \cite{?}. Great progress has been made in treating certain cancers (e.g. breast cancer \cite{?}), partly as a result of extensive screening programmes and improvements in imaging techniques allowing the tumour to be caught early before it has metastasized to other parts of the body. As yet, it has proven difficult to attain equally promising results for other cancers like kidney cancer, largely because kidney tumours can often only be detected at a much later stage of development (indeed, they are often detected incidentally, whilst a patient is being treated for a separate illness), making them significantly harder to treat successfully \cite{?}. Nonetheless, there continues to be significant research into kidney cancer: one of the most high-profile examples of this in recent years has been the development of new drugs such as Sutent (sunitinib), Avastin (bevacizumab), Nexavar (sorafenib) and Torisel (temsirolimus) \cite{?}.

In taking crucial decisions about how best to apply both these new treatments, and the treatments that were previously available, to kidney cancer patients, doctors need as much relevant information as possible. This information can often be derived from medical images (in the case of kidney cancer, this tends to mean CT or MRI), but extracting useful information from the vast amount of data generated by modern medical scanners can be a difficult task.