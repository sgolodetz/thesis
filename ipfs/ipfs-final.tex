\chapter{Principles of Partition Forests}
\label{chap:ipfs}

%---
\section{Chapter Overview}

In Chapter~\ref{chap:methodology}, I discussed the goals of my doctorate and the methods I chose to try to achieve them. This chapter introduces partition forests as a data structure, along with a set of novel algorithms for working with them. This lays the foundation for the two following chapters, which describe how partition forests can be constructed from images (Chapter~\ref{chap:segmentation}) and then used to identify features therein (Chapter~\ref{chap:featureid}). In that context, we will talk of partition forests as image partition forests (or IPFs).

As will be discussed in Chapter~\ref{chap:assessment}, however, the partition forest data structure and its accompanying algorithms are useful in a wider context than just imaging. In particular, we will later see another application of partition forests in the domain of hierarchical pathfinding.

%---
\section{What is a Partition Forest?}
%\label{sec:ipfs-description}

\subsection{Concept}

A partition forest is in essence a hierarchy of adjacency graphs that all partition the same object (the sense in which a graph can partition an object will be formalised in what follows). The object itself can be anything that can be divided into pieces, whether that be an image, a road network, an organisation, or even a pizza. Partition forests as a data structure are similar to the picture trees described in \cite{andrade03}. They are also related to the semantic segmentation trees found in \cite{al-haj08} and the binary partition trees found in \cite{salembier00}. The partition forest algorithms I will describe in \S\ref{sec:ipfs-algorithms}, however, are entirely novel.

The ideas behind partition forests can be most simply illustrated using the aforementioned example of a pizza (see Figure~\ref{fig:ipfs-description-pizza}). We consider three different levels of partitioning for the pizza: into slices, into portions, and into halves (e.g.~a vegetarian half and a meat half). Each half contains one or more portions and each portion contains one or more slices. All the slices taken together form the whole pizza, and none of them mutually overlap. This is equally true for both the portions and the halves. Taken together, the slices, portions and halves (the `pieces') form a hierarchy of partitions of the pizza, in which the individual pieces become nodes. (Note that the top-most level of the hierarchy is allowed to contain more than one node -- this is what makes the hierarchy a forest rather than a tree.) Each level of the hierarchy is an adjacency graph: in this case, the adjacency graphs for the slices and portions form rings (each slice/portion is adjacent to the slices/portions on either side of it), whilst the adjacency graph for the halves is just a single edge between them. The weights on the edges are not especially important in the case of pizzas (we could perhaps imagine them as indicating the ease of pulling the pizza apart at that point, if we wanted to push the analogy), but they are very important in the imaging domain, as we will see later.

%---
% TODO: fig:ipfs-description-pizza
\stufigex{width=15cm, height=24cm}{todo.png}{TODO}{fig:ipfs-description-pizza}{p}
%---

It is rare for pizza slices and portions to all be identical -- some slices will generally end up with more topping on than others, and some portions may contain more slices than others. When identifying an appropriate portion to choose, therefore, it is important to have relevant properties of the slices and portions available in order to make an informed choice. For that purpose, we can attach properties to the nodes of the forest that provide this information for later use. These properties can be different for different node layers. We can imagine maintaining a topping quality property for slices, with perhaps slice count and best/average topping quality properties for portions. What is important is that the properties of each node can be calculated directly from those of its children in the hierarchy. In this case, that would mean that the properties of a portion should be a function of only the properties of the slices in that portion.

\subsection{Definition}

Abstracting away from the intuitive description of partition forests just given, it is possible to define them more formally as follows:

\begin{definition}
An \textbf{object} is a non-empty set of basic components which together form a contiguous whole. (For example, a contiguous image region would be a non-empty set of pixels.)
\end{definition}

\begin{definition}
A set of k objects $\{o'_1,\ldots,o'_k\}$ \textbf{partitions} an object $o$ iff $\bigcup_i o'_i = o$ and $\forall i,j \cdot o'_i \cap o'_j = \emptyset$. We write the relation as $\mathcal{P}(\{o'_1,\ldots,o'_k\}, o)$.
\end{definition}

\begin{definition}
Given an object $o$, and two objects sets $O'_f = \{o'_{f1},\ldots,o'_{fk_f}\}$ and $O'_c = \{o'_{c1},\ldots,o'_{ck_c}\}$, satisfying $\mathcal{P}(O'_f,o)$ and $\mathcal{P}(O'_c,o)$, we say that $O'_c$ is a \textbf{coarser partition} of $o$ than $O'_f$ (written $O'_f \sqsubseteq O'_c$) iff for every object $o'_{ci} \in O'_c$ there exists a subset $S_i \subseteq O'_f$ such that $\mathcal{P}(S_i,o'_{ci})$. (In other words, $O'_f$ is a partition of $o$ in which each individual object in $O'_c$ has itself been partitioned.)
\end{definition}

\begin{definition}
Letting $\mbox{adj}_o(o'_i, o'_j)$ denote that sub-objects $o'_i$ and $o'_j$ are (in some sense) adjacent in an object $o$, we define a \textbf{weight function} $w_o$ for $o$ to be a function of type $\mathbb{P}(o) \times \mathbb{P}(o) \to \mathbb{R}^+$ that satisfies the following two requirements:
%
\begin{enumerate}

\item $w_o(o'_i, o'_j) \ne \infty$ when, and only when, $adj_o(o'_i, o'_j)$ is true

\item Given any sets $S_i$ and $S_j$ satisfying $\mathcal{P}(S_i,o'_i)$ and $\mathcal{P}(S_j,o'_j)$, the value $w_o(o'_i, o'_j)$ is a function of only the values in the set $\{w_o(s_i, s_j) \; | \; s_i \leftarrow S_i, \; s_j \leftarrow S_j\}$.

\end{enumerate}

\end{definition}

\begin{definition}
A \textbf{property set} is an ordered set (a tuple) of properties, each of which is a function that maps an object to a value (the types of the values may differ). For example, in the context of imaging it would be possible to have an area property that calculates the area of a given image region in pixels.
\end{definition}

\begin{definition}
Given a property set $P = (p_1,\ldots,p_k)$ and an object $o$, the \textbf{property value set} $V_P(o)$ is the ordered set that results from applying each property in $P$ to the object $o$, namely $(p_1(o),\ldots,p_k(o))$.
\end{definition}

\begin{definition}
We call a property set $P$ \textbf{directly calculable} from a property set $P'$ iff, for any given set of sub-objects $O'$ and object $o$ satisfying $\mathcal{P}(O',o)$, the property value set $V_P(o)$ is a function of only the property value sets in $\{V_{P'}(o') \; | \; o' \in O'\}$. We write this relation as $P' \hookrightarrow P$.
\end{definition}

\begin{definition}
A \textbf{partition node} is a node in a partition forest. Each node $n$ represents a given object, denoted as $\mbox{obj}(n)$. The set of objects represented by a node set $N$ can be denoted as $\mbox{Objs}(N)$.
\end{definition}

\begin{definition}
A \textbf{partitioning graph} $G(N,w_o,P)$ of an object $o$ is an undirected graph with weighted edges and property values on each node. It has ordered node set $N$, satisfying $\mathcal{P}(\mbox{Objs}(N),o)$, edge set $E = \{(n_i,n_j,w(\mbox{obj}(n_i),\mbox{obj}(n_j))) \; | \; n_i, n_j \leftarrow N\}$, and property value set tuple $\textit{VS} = (V_P(\mbox{obj}(n)) \; | \; n \leftarrow N)$.
\end{definition}

\begin{definition}
Given:

%-
\begin{enumerate}

\item An object $o$
\item A non-empty tuple $\textit{NS} = (N_1,\ldots,N_k)$, where:

%--
\begin{enumerate}

\item $\forall N_i \in \textit{NS} \cdot \mathcal{P}(\mbox{Objs}(N_i),o)$
\item $\mbox{Objs}(N_1) \sqsubseteq \ldots \sqsubseteq \mbox{Objs}(N_k)$ 
\item $\forall n \in N_1 \cdot |\mbox{obj}(n)| = 1$

\end{enumerate}
%--

\item A weight function $w_o$ for $o$
\item A non-empty tuple $\textit{PS} = (P_1,\ldots,P_k)$ satisfying $P_1 \hookrightarrow \ldots \hookrightarrow P_k$

\end{enumerate}
%-

\noindent We define the \textbf{partition forest} $PF_{\textit{NS},w_o,\textit{PS}}(o)$ to be the pair $(\textit{FL},\textit{PG})$, in which:

\begin{itemize}

\item $\textit{FL}$ is a set of forest links, defined as:
%
\[
\{(n_c,n_p) \; | \; n_c, n_p \leftarrow N_1,\ldots,N_k \mbox{ and } n_c \ne n_p \mbox{ and } \mbox{obj}(n_c) \subseteq \mbox{obj}(n_p)\}
\]

\item $\textit{PG}$ is an ordered set of partitioning graphs of $o$, defined as:
%
\[
(G(N_1,w_o,P_1),\ldots,G(N_k,w_o,P_k))
\]

\end{itemize}

\end{definition}

\noindent We can also define a parent/child relation between nodes, namely that $p = \mbox{parent}(c)$ iff $(c,p) \in \textit{FL}$. (This is equivalent to saying $c \in \mbox{children}(p)$.)

%---
\section{Partition Forest Algorithms}
\label{sec:ipfs-algorithms}

\subsection{Overview}

Having formally defined partition forests, I will now present a series of algorithms for working with them. The algorithms I have developed can be partitioned into two major classes: modification algorithms, which allow users to alter the structure of a forest, and selection algorithms, which provide a means of selecting nodes in the forest. The organisation of the algorithms is illustrated in Figure~\ref{fig:ipfs-algorithms-organisation}.

%---
\stufigex{width=15cm}{ipfs/ipfs-algorithms-organisation.png}{The organisation of the partition forest algorithms -- the red lines indicate the dependencies between the different algorithms.}{fig:ipfs-algorithms-organisation}{p}
%---

\subsection{Modification Algorithms}

\subsubsection{Overview}

The algorithms for modifying partition forests can be divided into four groups (in three levels). The lowest level contains four core algorithms: layer cloning, layer deletion, node splitting and sibling node merging (merging two or more nodes which share the same parent). These are all that is needed to allow the user to manually create any partition forest from a leaf layer of basic components (e.g.~pixels, in the case of images).\footnote{A good way of thinking about it (coming from a programming perspective) is that if you were to implement a partition forest class in an object-oriented programming language, the core algorithms would be the ones which had to be implemented as member functions of the class (in other words, they form the class's minimal interface).} More sophisticated algorithms can then be built on this foundation. The level above the core contains the zipping algorithms. These are essentially multi-layer split and merge operations -- instead of splitting or merging individual nodes, they split and merge branches of the hierarchy. They form the basis of algorithms such as non-sibling node merging and feature identification in the highest level.

\subsubsection{Core Algorithms}

\paragraph{Layer Cloning}

\begin{itemize}

\item \emph{Interface}. \texttt{clone_above_layer(layer)} and \texttt{clone_below_layer(layer)}

\item \emph{Description}. New layers can be inserted anywhere in the hierarchy by cloning a layer above or below the insertion point (for example, see Figure~\ref{fig:ipfs-algorithms-layercloning}). A partition forest is guaranteed to have at least one layer at all times, so there will always be an existing layer to clone. In terms of the earlier partition forest definition, this has the effect of inserting a copy of the partitioning graph into the set $\textit{PG}$, and adding to $\textit{FL}$ the appropriate forest links between nodes in the inserted layer and those in the layer(s) adjacent to it.

\item \emph{Usage}. Used to create additional layers that can later be modified using split/merge operations.

\item \emph{Complexity}. Let $n_\ell$ be the number of nodes and $e_\ell$ be the number of adjacency graph edges in layer $\ell$. Suppose layer $i$ is to be inserted between two existing layers, $b$ (below) and $a$ (above). Then, since we are cloning one of the two existing layers, either $n_i = n_b$ and $e_i = e_b$, or $n_i = n_a$ and $e_i = e_a$. In either case, the complexity of cloning the partitioning graph is $\Theta(e_i)$. Observe in passing that $n_i \in O(e_i)$, since the adjacency graph is connected, so there is no need to include $n_i$ when analysing the complexity of the cloning process. Forest links must then be added between layers $b$ and $i$, and layers $i$ and $a$. There are $\Theta(n_b)$ of the former, and $\Theta(n_i)$ of the latter (consider connecting each of the nodes in the two layers to its parent in the layer above). So the cost of adding the links is $\Theta(n_b + n_i)$. This is $\Theta(n_b)$, since $n_b \ge n_i$. The overall complexity is thus $\Theta(e_i + n_b)$. (If we are cloning the layer below, whether because there is no layer above or otherwise, this simplifies to $\Theta(e_b)$. If we are cloning the layer above, and there is a layer below, it becomes $\Theta(e_a + n_b)$. If we are cloning the layer above and there is no layer below, this simplifies to $\Theta(e_a)$.)

\end{itemize}

% TODO: fig:ipfs-algorithms-layercloning

\paragraph{Layer Deletion}

\begin{itemize}

\item \emph{Interface}. \texttt{delete_layer(layer)}

\item \emph{Description}. Any layer except the lowest can be deleted from the hierarchy (for example, see Figure~\ref{fig:ipfs-algorithms-layerdeletion}). In terms of the partition forest definition, this has the effect of removing both the specified partitioning graph from $\textit{PG}$ and the forest links referencing the deleted layer from $\textit{FL}$, and adding new forest links where necessary between any layers on either side of the one being removed.

\item \emph{Usage}. Used to delete a layer -- perhaps because it contains no objects of semantic interest.

\item \emph{Complexity}. Suppose layer $d$ is to be deleted and that it lies between two other layers, $b$ (below) and $a$ (above). The complexity of removing the partitioning graph from $\textit{PG}$ is $\Theta(1)$. The forest links between layers $b$ and $d$, and layers $d$ and $a$ must then be deleted. This has complexity $\Theta(n_b + n_d)$, which simplifies to $\Theta(n_b)$ since $n_b \ge n_d$. Finally, forest links must be added between layers $b$ and $a$. This has complexity $\Theta(n_b)$. The overall complexity of the whole algorithm is thus $\Theta(n_b)$. (If, instead of there being layers above and below that being deleted, there is only a layer below, the complexity is still $\Theta(n_b)$.)

\end{itemize}

% TODO: fig:ipfs-algorithms-layerdeletion

\paragraph{Node Splitting}

\begin{itemize}

\item \emph{Interface}. \texttt{split_node(node, $\{group_1,\ldots,group_n\}$)}

\item \emph{Description}. Nodes representing objects containing more than one basic component (e.g.~a pixel) can be split into multiple nodes representing smaller objects. (Note that the definition of objects requires that each of these smaller objects must be contiguous.) The algorithm takes as input the node to be split, and a set of groups of the node's children into which to split it, where each group is a set of child nodes (see Figure~\ref{fig:ipfs-algorithms-nodesplitting} for an example). The process then involves the following steps:
%
\begin{enumerate}

\item Check the precondition that the node to be split represents an object containing more than one basic component.

\item Check the precondition that each group containing more than one child node is contiguous. This can be done with a simple flooding algorithm on the partitioning graph of the child layer (the layer containing the nodes in the groups). Starting from the first node in the group, recursively visit adjacent nodes that are in the group and have not yet been seen. The process terminates when either the entire group has been visited (in which case the group is contiguous), or when there are no more nodes to which to recurse and not every node in the group has yet been visited (in which case the group is not contiguous).

\item Remove the forest links between the node being split (hereafter called the `initial parent') and its children.

\item Create new nodes in the parent layer (the one containing the initial parent) as necessary for all but one of the groups (we reuse the initial parent for one of them) and add forest links between the new nodes and the parent of the node being split (the initial grandparent), if any.

\item Link the child nodes in each group to their new parents and calculate the property value sets of the group parents from those of their children in each case.

\item Locally rebuild the partitioning graph for the parent layer: this involves first removing the edges for the initial parent, some of which will no longer be valid, and then calculating the edges connected to all the group parents using the partitioning graph for the child layer. Note that it is at this point that the second requirement on the weight function for the partition forest becomes relevant: it must be possible to calculate the weight on the edge between two nodes in the parent layer from only the weights on the edges between their respective children in the child layer. How that actually happens depends on how a specific application chooses to define its weight function. In the case of the `lowest pass point' weight function we will see in Chapter~\ref{chap:segmentation}, the weight on the edge between two parent nodes is simply the minimum of the weights on the edges between their respective children, but any number of other weight functions are possible.

\end{enumerate}

\item \emph{Usage}. TODO

\item \emph{Complexity}. TODO

\end{itemize}

\paragraph{Sibling Node Merging}

TODO

\subsubsection{Zipping Algorithms}

\paragraph{Unzipping}

TODO

\paragraph{Zipping}

TODO

\subsubsection{Higher-Level Algorithms}

\paragraph{Non-Sibling Node Merging}

TODO

\paragraph{Parent Switching}

TODO

\subsubsection{Applied Algorithms}

\paragraph{Feature Identification}

TODO

\subsection{Selection Algorithms}

\subsubsection{Overview}

TODO

\subsubsection{Add To Selection}

TODO

\subsubsection{Remove From Selection}

TODO

\subsubsection{View Selection At Layer}

TODO

%---
\section{Chapter Summary}

TODO
