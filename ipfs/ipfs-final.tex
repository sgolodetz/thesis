\chapter{Partition Forests}
\label{chap:ipfs}

%---
\section{Chapter Overview}

In Chapter~\ref{chap:methodology}, I discussed the goals of my doctorate and the methods I chose to try to achieve them. This chapter introduces partition forests as a hierarchical representation for aggregate objects in general, and images in particular. I present a hierarchy of novel editing algorithms for partition forests, and propose partition forest selection and partition forest multi-feature selection data structures to support subtree-based selection and identification of nodes within forests. I further show how these features can be integrated into a graphical user interface for working with image partition forests (that is, partition forests in an imaging context). This lays the foundation for the following two chapters, which describe how partition forests can be constructed from images (Chapter~\ref{chap:segmentation}) and then used to identify features therein (Chapter~\ref{chap:featureid}).

%---
\section{What is a Partition Forest?}

\subsection{Concept}

A partition forest is in essence a hierarchy of adjacency graphs that all partition the same object (the sense in which a graph can partition an object will be formalised in \S\ref{sec:ipfs-definition}). The object itself can be anything that can be divided into pieces, whether that be an image, a road network or an organisation. As was seen in \S\ref{sec:background-partitionhierarchies}, partition forests (whether called by that name or otherwise) have been widely used as a hierarchical representation for images. However, surprisingly little attention has been devoted to how to edit them post-construction (with perhaps the notable exception being Nacken's work in \cite{nacken95}).

\newpage

The key components that make up a partition forest are illustrated in Figure~\ref{fig:ipfs-concept}, which shows a partition forest that might be constructed for a simple $4 \times 4$ image. A forest is made up of a number of layers, each of which is an adjacency graph representing a partition of an object (in this case, the image). Each partition is refined by the next highest partition, in the sense that each node in partition $i+1$ is the union of some of the nodes in partition $i$.

The nodes in each layer represent groups of the smallest sub-objects into which the represented object can be divided (in this case, each node represents an image region, consisting of a group of pixels), and can have layer-dependent properties associated with them (shown as blue text in the figure). Properties of nodes in the leaf layer can be assigned arbitrarily, but properties of nodes in branch layers must be functions of the properties of their children in the layer below (this will be defined formally in \S\ref{sec:ipfs-definition}). In this example, a single, arbitrary value has been associated with each node in the leaf layer of the forest. Nodes in higher layers have been given a `mean value' property that is calculated from the values of the subsumed leaf nodes.

Each layer also contains edges between nodes that are in some sense adjacent (in the case of images, this is defined in such a way that the nodes are considered adjacent if their corresponding regions are adjacent in the image). Each edge has an associated value (shown as underlined text in the figure). The values on the edges in the leaf layer can be assigned according to any scheme desired -- in this example, they represent the height of the `lowest pass point' between adjacent nodes, based on the values associated with the pixels. The value on an edge between a pair of nodes in a branch layer must be a function of the values on any edges between their respective children in the layer below (again, this will be defined formally in \S\ref{sec:ipfs-definition}). In this example, the value on an edge between two nodes, $u$ and $v$, in a branch layer is calculated to be the smallest value on any edge between a child of $u$ and a child of $v$, in keeping with the lowest pass idea above.

In addition to the forest layers themselves, a partition forest also contains forest links that join the nodes in adjacent layers together (the coloured, dashed lines in the figure). In particular, there is a link between each node and the node that contains it in the layer above. These links naturally define parent/child relationships between forest nodes.

%---
\stufigex{height=24cm}{ipfs/ipfs-concept.png}{The concept of a partition forest (see main text for discussion)}{fig:ipfs-concept}{p}
%---

\subsection{Definition}
\label{sec:ipfs-definition}

It is possible to define partition forests more formally as follows:

\begin{definition}
An \textbf{object} is a non-empty set of basic components which together form a contiguous whole. (For example, a contiguous image region would be a non-empty set of pixels.)
\end{definition}

\begin{definition}
A set of k objects $\{o'_1,\ldots,o'_k\}$ \textbf{partitions} an object $o$ iff $\bigcup_i o'_i = o$ and $\forall i,j \cdot o'_i \cap o'_j = \emptyset$. We write the relation as $\mathcal{P}(\{o'_1,\ldots,o'_k\}, o)$.
\end{definition}

\begin{definition}
Given an object $o$, and two objects sets $O'_f = \{o'_{f1},\ldots,o'_{fk_f}\}$ and $O'_c = \{o'_{c1},\ldots,o'_{ck_c}\}$, satisfying $\mathcal{P}(O'_f,o)$ and $\mathcal{P}(O'_c,o)$, we say that $O'_c$ is a \textbf{coarser partition} of $o$ than $O'_f$ (written $O'_f \sqsubseteq O'_c$) iff for every object $o'_{ci} \in O'_c$ there exists a subset $S_i \subseteq O'_f$ such that $\mathcal{P}(S_i,o'_{ci})$. (In other words, $O'_f$ is a partition of $o$ in which each individual object in $O'_c$ has itself been partitioned.)
\end{definition}

\begin{definition}
Letting $\mbox{adj}_o(o'_i, o'_j)$ denote that sub-objects $o'_i$ and $o'_j$ are (in some sense) adjacent in an object $o$, we define a \textbf{weight function} $w_o$ for $o$ to be a function of type $\mathbb{P}(o) \times \mathbb{P}(o) \to \mathbb{R}^+$ that satisfies the following two requirements:
%
\begin{enumerate}

\item $w_o(o'_i, o'_j) \ne \infty$ when, and only when, $adj_o(o'_i, o'_j)$ is true

\item Given any sets $S_i$ and $S_j$ satisfying $\mathcal{P}(S_i,o'_i)$ and $\mathcal{P}(S_j,o'_j)$, the value $w_o(o'_i, o'_j)$ is a function of only the values in the set $\{w_o(s_i, s_j) \; | \; s_i \in S_i, \; s_j \in S_j\}$.

\end{enumerate}

\end{definition}

\begin{definition}
A \textbf{property set} is an ordered set (a tuple) of properties, each of which is a function that maps an object to a value (the types of the values may differ). For example, in the context of imaging it would be possible to have an area property that calculates the area of a given image region in pixels.
\end{definition}

\begin{definition}
Given a property set $P = (p_1,\ldots,p_k)$ and an object $o$, the \textbf{property value set} $V_P(o)$ is the ordered set that results from applying each property in $P$ to the object $o$, namely $(p_1(o),\ldots,p_k(o))$.
\end{definition}

\begin{definition}
We call a property set $P$ \textbf{directly calculable} from a property set $P'$ iff, for any given set of sub-objects $O'$ and object $o$ satisfying $\mathcal{P}(O',o)$, the property value set $V_P(o)$ is a function of only the property value sets in $\{V_{P'}(o') \; | \; o' \in O'\}$. We write this relation as $P' \hookrightarrow P$.
\end{definition}

\begin{definition}
A \textbf{partition node} is a node in a partition forest. Each node $n$ represents a given object, denoted as $\mbox{obj}(n)$. The set of objects represented by a node set $N$ can be denoted as $\mbox{Objs}(N)$.
\end{definition}

\begin{definition}
A \textbf{partitioning graph} $G(N,w_o,P)$ of an object $o$ is an undirected graph with weighted edges and property values on each node. It has ordered node set $N$, satisfying $\mathcal{P}(\mbox{Objs}(N),o)$, edge set $E = \{(\{n_i,n_j\},w(\mbox{obj}(n_i),\mbox{obj}(n_j))) \; | \; n_i, n_j \in N \mbox{ and } n_i \ne n_j\}$, and property value set tuple $\textit{VS} = (V_P(\mbox{obj}(n)) \; | \; n \in N)$.
\end{definition}

\begin{definition}
Given:

%-
\begin{enumerate}

\item An object $o$
\item A non-empty tuple $\textit{NS} = (N_1,\ldots,N_k)$, where:

%--
\begin{enumerate}

\item $\forall N_i \in \textit{NS} \cdot \mathcal{P}(\mbox{Objs}(N_i),o)$
\item $\mbox{Objs}(N_1) \sqsubseteq \ldots \sqsubseteq \mbox{Objs}(N_k)$ 
\item $\forall n \in N_1 \cdot |\mbox{obj}(n)| = 1$

\end{enumerate}
%--

\item A weight function $w_o$ for $o$
\item A non-empty tuple $\textit{PS} = (P_1,\ldots,P_k)$ satisfying $P_1 \hookrightarrow \ldots \hookrightarrow P_k$

\end{enumerate}
%-

\noindent We define the \textbf{partition forest} $PF_{\textit{NS},w_o,\textit{PS}}(o)$ to be the pair $(\textit{FL},\textit{PG})$, in which:

\begin{itemize}

\item $\textit{FL}$ is a set of forest links, defined as:
%
\[
\{(n_c,n_p) \; | \; \exists i \in [1,k-1] \cdot n_c \in N_i \mbox { and } n_p \in N_{i+1} \mbox{ and } \mbox{obj}(n_c) \subseteq \mbox{obj}(n_p)\}
\]

\item $\textit{PG}$ is an ordered set of partitioning graphs of $o$, defined as:
%
\[
(G(N_1,w_o,P_1),\ldots,G(N_k,w_o,P_k))
\]

\end{itemize}

\end{definition}

\noindent We can also define a parent/child relation between nodes, namely that $p = \mbox{parent}(c)$ iff $(c,p) \in \textit{FL}$. (This is equivalent to saying $c \in \mbox{children}(p)$.)

%---
\section{Mutating Algorithms for Partition Forests}

Partition forests, as presented thus far, are a useful hierarchical representation for aggregate objects, but they are \emph{static}: that is, once a partition forest has been constructed for a given object, it does not change. This can be problematic, because constructing a perfect forest is potentially hard, and the number of possible forests for a given object is (at least theoretically) infinite. Even being more practical, and imposing the reasonable condition that every edge in the adjacency graph for the forest's leaf layer should have been elided (merging the two nodes it joins) by branch layer $E$, where $E$ is the total number of edges in the leaf adjacency graph, we can observe that there are still $E^2$ possible forests for an object (the $E^2$ comes from choosing a layer between $1$ and $E$ inclusive at which each leaf layer edge is removed by merging the nodes it joins). Putting this into context, that means that for a $4$-connected image of size $r \times c$, there are $(4rc - 2(r+c))^2$ possible partition forests (over a million possibilities for a $512 \times 512$ image). There is therefore a pressing need for mutating algorithms that allow us to transform one partition forest into another, in order to allow a user to work around potential issues with the initial partition forest construction. In this section, I present a hierarchy of novel algorithms (see Figure~\ref{?}) that allow users to edit partition forests, thereby changing them from being merely a static data structure into being a dynamic one.

\subsection{Core Algorithms}

In order to be able to transform a partition forest for an object $o$ into any other partition forest over $o$, a certain minimal set of forest operations must be possible. Specifically, it must be possible to clone and delete forest layers, and merge sibling forest nodes (nodes that share the same parent). To show that these are both necessary and sufficient, we first prove the following theorem:

\begin{theorem}
\label{thrm:ipfs-construction}
It is possible to construct any partition forest for a particular object by starting from the leaf layer and using only clone layer and merge sibling nodes operations. They are also the absolute minimum necessary.
\end{theorem}

\begin{proof}
(Sufficiency) Every branch node in the forest is the union of some of the nodes in the layer below (its children). To form each new layer of the forest, it therefore suffices to clone the current topmost layer (using clone layer) and merge the nodes which comprise each branch node (using merge sibling nodes). This can be repeated as many times as necessary to form the desired forest.
\end{proof}

\begin{proof}
(Necessity) Without the clone layer operation (or an equivalent way of creating new layers), it is impossible to create partition forests with more than one layer. Without the merge sibling nodes operation (or an equivalent way of merging nodes in a given layer), it is impossible to create non-trivial branch nodes.\qed
\end{proof}

\noindent Given this, it is simple to extend it to transformations between arbitrary partition forests over the same object:

\begin{theorem}
\label{thrm:ipfs-transformation}
It is possible to transform any partition forest $F$ for a particular object into any other partition forest $F'$ for the same object using only clone layer, delete layer and merge sibling nodes operations. They are also the absolute minimum necessary.
\end{theorem}

\begin{proof}
(Sufficiency) Trivial: it suffices to delete all the branch layers in $F$ and apply Theorem~\ref{thrm:ipfs-construction} to construct $F'$.
\end{proof}

\begin{proof}
(Necessity) Without the clone layer operation, it is impossible to transform to a forest with more layers than currently present. Without the delete layer operation, it is impossible to transform to a forest with fewer layers than currently present. Without the merge sibling nodes operation, it is impossible to alter any of the layers. All three operations are thus individually necessary.\qed
\end{proof}

\noindent Theorem~\ref{thrm:ipfs-transformation} proves that only three core operations are strictly necessary to allow partition forests to be edited arbitrarily. However, in practice it is vitally important to also directly support node splitting as a core operation if efficient editing is desired (it is technically possible to split a node by reconstructing its entire layer, but this is clearly impractical). For this reason, the core partition forest algorithms are defined as (1) layer cloning, (2) layer deletion, (3) sibling node merging and (4) node splitting. Each of these operations is now examined in detail.

\subsubsection{Layer Cloning}

TODO