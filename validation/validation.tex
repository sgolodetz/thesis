\chapter{Validation}
\label{chap:validation}

%##################################################################################################
\section{Chapter Overview}
%##################################################################################################

%In the previous chapter, I presented a set of novel feature identifiers that can be used to extract features such as the spine, spinal cord, ribs, aorta and soft-tissue organs from abdominal CT scans in an automatic way. It is clearly the case, however, that some of these feature identifiers are more robust than others (spine identification, for example, can be expected to be substantially more robust than liver identification). In this chapter, I seek to quantify the effectiveness of the identifiers by validating them against known results (a `gold standard'). This provides the basis for the following chapter, in which I critically assess my approach as a whole in the context of existing work in this area.

In order to evaluate the effectiveness of my approach, I undertook two separate pieces of validation work. The first piece involved quantifying the accuracy of the 3D feature identifiers presented in the previous chapter -- this was done by comparing their output to `gold standard' results produced manually in collaboration with a radiologist. The second piece involved validating the accuracy of my volume calculations -- this was done by manually identifying the heart in a large number of series and using my program to calculate its volume in each case, then comparing the results to known volumes provided by the radiologist. Although the focus of my work was on \emph{abdominal} CT scans, the volume calculation process I implemented is independent of body location: if it is accurate in the case of the heart (for which in this case ample volume results were available), then it is also necessarily accurate for organs in the abdomen.

TODO

%##################################################################################################
\section{Validation of Feature Identifiers}
%##################################################################################################

TODO

%################################################
\subsection{Production of a Gold Standard}
%################################################

TODO

%################################################
\subsection{Results}
%################################################

TODO

%##################################################################################################
\section{Validation of Volume Calculations}
%##################################################################################################

TODO

%##################################################################################################
\section{Chapter Summary}
%##################################################################################################

TODO
