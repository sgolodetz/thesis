\chapter{Segmentation}
\label{chap:segmentation}

\iffalse

- Quick summary of watershed/waterfall methods from Chapter 2
- Explanation that this chapter will extend the original explanation to include image partition forests + definition of them
- Overview of the implementation algorithms used (incl. the preprocessing necessary)
- Detailed implementation of the preprocessing steps
- Detailed implementation of the watershed transform (reference original paper)
- Detailed implementation of the waterfall algorithm (reference original paper)
- Demonstration that the produced segmentations are generally good, whilst showing that they aren't always
- Discussion of factors affecting the segmentation (CT window, image resolution and quality, etc.)
- 3D visualizations to illustrate that good results can be obtained through automatic segmentation followed by manual feature identification (explain user interface)

\fi

The essential ideas behind the watershed/waterfall method were described in Chapter~\ref{chap:background} but, as discussed there, it can be implemented in a number of different ways. Since I have not been able to find a full implementation which outputs a partition hierarchy described in detail anywhere in the relevant literature \cite{?}, I will describe my implementation (based on the ideas in \cite{meijster98} and \cite{marcotegui05}) in this chapter, before presenting some results and discussion.

\section{Image Partition Forests}

At the highest possible level of abstraction, the watershed/waterfall method can be seen to take an image (either 2D or 3D) as input, and produce an \emph{image partition forest} (IPF) for the image as output (see Figure~\ref{?}). A formal definition for image partition forests can be built up as follows:

\begin{definition}
A set of $k$ objects $\{o_1,\ldots,o_k\}$ \textbf{partitions} object $o'$ iff $\bigcup_i o_i = o'$ and $\forall i,j \cdot i \ne j \Rightarrow o_i \cap o_j = \emptyset$. We write the relation as $\mathbf{P}(\{o_1,\ldots,o_k\},o')$.
\end{definition}

\begin{definition}
Given two object sets $O_f = \{o_{f1},\ldots,o_{fk(f)}\}$ and $O_c = \{o_{c1},\ldots,o_{ck(c)}\}$, and an object $o'$ partitioned by both $O_f$ and $O_c$, we say that $O_c$ is a \textbf{coarser partition} of $o'$ than $O_f$ (written $O_f \sqsubseteq O_c$) iff for every object $o_{ci} \in O_c$ there exists a subset $S_i \subseteq O_f$ such that $\mathbf{P}(S_i, o_{ci})$, and furthermore, $\forall i,j \in [1,k(c)] \cdot i \ne j \Rightarrow S_i \cap S_j = \emptyset$. (In other words, $O_f$ is a partition of $o'$ in which each individual object in $O_c$ has itself been partitioned.)
\end{definition}

\begin{definition}
Given a set of regions in an image $I$, $R_I = \{r_1,\ldots,r_k\}$, a function $w : R_I \times R_I \rightarrow \mathbb{R}^{+}$ is a \textbf{weight function} for $R_I$ iff $w(r_i,r_j) \ne \infty$ when, and only when, $r_i$ and $r_j$ are physically adjacent (share a common border) in $I$.
\end{definition}

\begin{definition}
A \textbf{weight function generator} is a function of type $(R : \{\mbox{Region}\}) \rightarrow (R \times R \rightarrow \mathbb{R}^{+})$.
\end{definition}

\begin{definition}
The \textbf{region adjacency graph} $\mbox{RAG}_{\mbox{WFG}}(R_I)$ is the graph with nodes $r_i$ and edge weights given by the weight function $\mbox{WFG}(R_I)$.
\end{definition}

\begin{definition}
Let $T$ be a technique for generating image partition forests, and $\mbox{WFG}_T$ be its associated weight function generator. Given an image $I$, the \textbf{image partition forest} $\mbox{IPF}_{T}(I)$ is the sequence of $n_T$ pairs $[(P_{T1},\mbox{RAG}_{\mbox{WFG}_T}(P_{T1})),\ldots,(P_{Tn_T},\mbox{RAG}_{\mbox{WFG}_T}(P_{Tn_T}))]$, where $P_{Ti}$ is the $i^{th}$ partition produced by technique $T$ and satisfies $\mathbf{P}(P_{Ti},I)$, and $\forall i \in [1,n_T) \cdot P_{Ti} \sqsubseteq P_{T(i+1)}$.
\end{definition}

This formal presentation of image partition forests is precise, but unfortunately rather opaque. A more intuitive description would be that an image partition forest is a sequence of partitions of the same image (accompanied by corresponding region adjacency graphs), each of which (aside from the initial, finest, partition) is formed by merging some of the regions in its immediately preceding partition together. For example, consider Figure~\ref{?}.