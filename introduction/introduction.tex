\chapter{Introduction}
\label{chap:introduction}

%The intertwined fields of segmentation and feature identification pose fundamental challenges in a variety of different domains. One such domain is that of image processing, in which the segmentation problem is first and foremost the task of partitioning an image into regions that correspond to `salient' features therein, although the term is also commonly used to refer to the related task of determining the boundary, or contour, around a feature of interest. The goal of feature identification, by contrast, is to assign semantic meaning to some or all of these regions.

In the most general sense, the problem of segmentation is the challenge of how to partition an object into pieces in such a way as to ensure that those pieces have some meaning in a given domain. The related (and to some extent overlapping) problem of feature identification is how to then ascribe meaning to some or all of those pieces. Whilst a huge variety of techniques have been developed for these problems, there are many domains in which it is difficult to specify, at least in a sufficiently precise manner, what constitutes a meaningful region and what does not. For that reason both problems remain major research challenges.

Nevertheless, an ability to solve these problems, even in an imperfect way, is often a crucial precursor to applying other algorithms in various domains, making them worthy targets for research efforts. For example, in the medical imaging domain, it is helpful to be able to translate a volume of data produced by a scanner into a 3D mesh, allowing doctors to visualize the state of a patient's organs in a more direct manner. Whilst there are existing algorithms to convert \emph{labelled} volume data to a 3D mesh (e.g.~\cite{wu03}), these rely on a priori knowledge of the location of the organs. That is, they rely on the organs having being labelled in the volume in advance. Producing such a labelling is a segmentation and feature identification problem.

This thesis dissertation is about tackling the segmentation and feature identification problems for a particular type of medical image, namely computerised tomography (CT) scans of a patient's abdomen. The first practical CT scanner was originally developed in 1971 by Sir Godfrey Hounsfield, working at the EMI Central Research Laboratories in Hayes, UK. CT scanners work using X-rays (see Figure~\ref{fig:introduction-ctscanning}) and produce images whose pixels are scalar values on a scale of radiodensity known (after Sir Godfrey) as the Hounsfield scale. They are used extensively in modern medical practice, both for diagnosing illness and for evaluating a patient's response to therapy.

My underlying interest is in renal (kidney) cancers, in the diagnosis of which CT scanners are generally used to produce images of the abdomen (see Figure~\ref{fig:introduction-kidneyslocation}). Whilst less prevalent than some other types of cancer -- notably breast, lung, colorectal and prostate -- renal cancers nevertheless kill thousands of people a year in the UK.\footnote{According to Cancer Research UK \cite{cruk-kidneycancermortality}, 3752 people died from a kidney cancer in the UK in 2007, the most recent year for which published statistics are currently available.} An ability to visualize the state of a renal cancer patient's abdominal organs in 3D can provide doctors with an extra tool when making important decisions about how best to treat the patient. In a similar vein, being able to calculate the volume of a renal tumour can help doctors track changes over time that may be relevant when evaluating a patient's response to therapy. For these, and other, applications, finding a way to segment and identify features in abdominal CT scans is a key prerequisite.

TODO
