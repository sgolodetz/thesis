\glossary{name={Auto-Transplantation},description={A surgical operation to transplant tissue (e.g.~an organ) from one part of a patient's body to another.}}

\glossary{name={centipede},description={The trial segmentation, feature identification and 3D visualization system built to support this thesis.}}

\glossary{name={Computerised Tomography (CT)},description={A medical imaging modality based on X-rays. Computerised tomography scanners produce Hounsfield images, which are greyscale images whose values lie on the \emph{Hounsfield scale} of radiodensity.}}

\glossary{name={GUI},description={An acronym for Graphical User Interface.}}

\glossary{name={Hounsfield Scale},description={A scale that quantifies the radiodensity of different tissue types in so-called Hounsfield units (HU). It runs from $-1024$ HU (least dense / air) to $3071$ (most dense).}}

\glossary{name={Image Partition Forest (IPF)},description={The specific form of \emph{partition forest} used for images. Each leaf node represents a single pixel in an image, and each branch node represents an image region.}}

\glossary{name={Magnetic Resonance Imaging (MRI)},description={A medical imaging modality based on powerful magnetic fields.}}

\glossary{name={millipede},description={The final segmentation, feature identification and 3D visualization system built to support this thesis.}}

\glossary{name={MST},description={An acronym for Minimum Spanning Tree.}}

\glossary{name={Partial Nephrectomy},description={A surgical operation to remove part (but not all) of a patient's kidney (from the Greek \emph{nephros} = kidney and \emph{-ectomy} = surgical removal of).}}

\glossary{name={Partition Forest},description={A nested hierarchy of graphs, each of which partitions the same object (e.g.~an image). See Chapter~\ref{chap:ipfs}.}}

\glossary{name={Region Adjacency Graph (RAG)},description={A (simple) weighted, undirected, graph expressing the adjacency relationships between a set of regions. Each region is represented by a node of the graph, and regions are said to be adjacent iff there is a graph edge joining their corresponding nodes. The weights on the edges are often used to express some notion of the degree of similarity between adjacent regions.}}

\glossary{name={Renal},description={Pertaining to the kidneys (from the Latin \emph{renes}, meaning kidneys).}}

\glossary{name={Renal Pelvis},description={The funnel-like part of the kidney that joins the kidney and the ureter.}}

\glossary{name={Transverse Processes},description={Bony structures that protrude to the left and right of vertebrae in the spine (every vertebra has two transverse processes, one on each side).}}

\printglossary
